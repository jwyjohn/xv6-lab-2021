\chapter{实验环境配置}
\begin{introduction}
    \item 虚拟机安装 Ubuntu 20.04
    \item 配置 RISC-V 相关工具链
    \item 获取原始的 xv6 实验代码
\end{introduction}

\section{虚拟机安装 Ubuntu 20.04}

虽然在目前的 Windows 10 中,微软提供了 Windows Subsystem for Linux ( WSL )的功能,但由于其兼容性和稳定性的问题,笔者仍然推荐使用传统的虚拟机方式安装整个 xv6 实验用的 Ubuntu 20.04。

笔者选择的是由 Oracle 开源的 Oracle VM VirtualBox 作为虚拟机的平台,其余平台,例如商用的 VMWare 等,其安装过程应当也是类似的。首先到官网下载 Oracle VM VirtualBox 的安装包,访问 \url{https://www.virtualbox.org/wiki/Downloads} 后,选取 Windows hosts 进行下载,然后使用默认设置进行安装,并重启计算机。

\begin{theorem}[硬件虚拟化 VT-x] 
    注意,使用虚拟机的 x86 兼容机需在 BIOS 中开启 VT-x 等虚拟化选项,一些硬件厂商在出厂时未默认开启,故需要自行查找相关文档进行设置。
\end{theorem}

安装完成后,下载我们需要的 Ubuntu 20.04 镜像。考虑到国内特殊的网络环境,不建议直接在官网上下载,而是从国内一些高校的镜像站中进行下载。例如,清华大学镜像站 TUNA 提供 Ubuntu 20.04 的镜像见脚注\footnote{\url{https://mirrors.tuna.tsinghua.edu.cn/ubuntu-releases/20.04/ubuntu-20.04.4-desktop-amd64.iso}} 。下载完成后,使用校验值计算工具计算其 SHA256 ,并与官网上提供的校验值进行对比,无误后即可继续使用。

在 Oracle VM VirtualBox 中新建虚拟机,选择系统为 Ubuntu ,然后使用默认设置一路向下,完成虚拟机的创建。然后在存储设置中,将光驱置入我们刚刚下载的镜像,然后启动虚拟机,按 Ubuntu 的提示进行安装,设置时区、账号、密码等。笔者为方便后续步骤,设置的账号为 osexp ,密码为 123456 。

\begin{theorem}[离线安装 Ubuntu] 
    注意,安装 Ubuntu 时若网络状况不佳,下载各类软件包的时间过长,需要断开虚拟机的网络连接,在离线的状态下进行安装。
\end{theorem}

Ubuntu 安装完成后,我们需要对其进行一些配置。首先是配置 Ubuntu 的软件源,使其使用国内镜像,以加快安装后续工具链的速度。Ubuntu 的软件源配置文件是 \lstinline{/etc/apt/sources.list}。将系统自带的该文件做个备份,将该文件替换为下面内容,即可使用 TUNA 的软件源镜像。
\begin{lstlisting}
deb https://mirrors.tuna.tsinghua.edu.cn/ubuntu/ focal main restricted universe multiverse
deb https://mirrors.tuna.tsinghua.edu.cn/ubuntu/ focal-updates main restricted universe multiverse
deb https://mirrors.tuna.tsinghua.edu.cn/ubuntu/ focal-backports main restricted universe multiverse
deb https://mirrors.tuna.tsinghua.edu.cn/ubuntu/ focal-security main restricted universe multiverse
\end{lstlisting}

配置完成后,以管理员权限执行 \lstinline{apt-get update} ,然后可以安装一些必须的软件。

\begin{proposition}[推荐的操作方式]
    由于虚拟机的性能限制,笔者不建议直接在虚拟机的桌面系统中进行开发操作,而是建议配置好 ssh 服务器和 VirtualBox 的端口映射,从主机使用 VS Code 等工具通过 ssh 对虚拟机进行操作。具体的配置方式请自行搜索。
\end{proposition}

\section{配置 RISC-V 相关工具链}

如果前文的镜像配置成功,那就可以直接在 Ubuntu 的终端中使用下面两条命令安装相关工具链:
\begin{lstlisting}
sudo apt-get update && sudo apt-get upgrade -y
sudo apt-get install -y git build-essential gdb-multiarch qemu-system-misc gcc-riscv64-linux-gnu binutils-riscv64-linux-gnu
\end{lstlisting}

输入管理员密码后,耐心等待即可安装成功,若中途出现问题,则可能需要从安装 Ubuntu 20.04 开始重试。

\section{获取原始的 xv6 实验代码}

首先,确认上述的工具链已经配置完成,各环境变量也都设置好。然后对 git 进行配置,主要在终端中使用下面的命令设置 git 的用户名和邮箱:
\begin{lstlisting}
git config --global user.name "jwy"
git config --global user.email "1951510@tongji.edu.cn"
\end{lstlisting}

设置完成后,使用 git 将原始的 xv6 代码库克隆下来:
\begin{lstlisting}
git clone git://g.csail.mit.edu/xv6-labs-2021
\end{lstlisting}