\chapter{序}

操作系统作为沟通软件与硬件的桥梁,其重要性无需多言。学习操作系统除了学习其理论外,真正“把玩”一个操作系统,可以加深对操作系统的理解,并且了解到理论与现实世界的差距。各个学校对于操作系统的教学方式各有区别,所使用的材料也各有特色。在诸多公认的优秀的操作系统的课程中, MIT 的 6.S081 可以说是经典的课程:该课程最经典的地方并不在于其讲课,而是在于其精心设计的 Labs 。6.S081 的 Labs 使用的是基于 Unix v6 改写的一个教学用操作系统,被称为 xv6 。这套基于 xv6 的实验几乎涵盖了操作系统各个方面的精髓,并且没有被工业实现上的繁琐的细节所困。

~\\

刚开始写 xv6 实验的时候,笔者深切地感受到开发操作系统的诸多困难:即便平日里习以为常的机制,在没有操作系统的支持下,实现起来就会困难许多。但是在逐渐适应了这个受限的、几乎是裸机的平台后,笔者逐渐能够利用有限的手段为 xv6 提供各种新的功能:例如增加系统调用、编写设备驱动、扩充文件系统等。而在进行了这些工作后进行回顾,笔者对 xv6 和其它操作系统也能看得更加透彻了。

~\\

有幸我校的操作系统课程设计课程可以选取 xv6 作为课题,故而笔者在这本实验报告中记录在2022年小学期期间完成全部 xv6 Labs 的过程。

~\\

姜文渊

2022年暑假
